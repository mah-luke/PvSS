\documentclass[conference]{IEEEtran}

% Packages
\usepackage{cite}
\usepackage{amsmath,amssymb,amsfonts}
\usepackage{algorithmic}
\usepackage{graphicx}
\usepackage{hyperref, url}
\usepackage{textcomp}
\usepackage{xcolor}
\def\BibTeX{{\rm B\kern-.05em{\sc i\kern-.025em b}\kern-.08em
   T\kern-.1667em\lower.7ex\hbox{E}\kern-.125emX}}
\begin{document}

\title{Summary of Game Agent ``Fun'' for the game Risk}

\author{
    \IEEEauthorblockN{Lukas Mahler}
    \IEEEauthorblockA{
        student id 11908553\\
        lukas.mahler@student.tuwien.ac.at
    }
    \and
    \IEEEauthorblockN{Frank Ebel}
    \IEEEauthorblockA{
        student id 01429282\\
        frank.ebel@student.tuwien.ac.at
    }
}

\maketitle

\begin{abstract}
This document is a summary of the Game Agent ``Fun'' for the game Risk.
It contains our heuristics and approach for Monte Carlo tree search (MCTS).
\end{abstract}

\begin{IEEEkeywords}
MCTS, Game agent,  Game Tree Search
\end{IEEEkeywords}

\section*{Repository}

Our source code can be found on \href{https://github.com/mah-luke/PvSS}{GitHub}.

\section{Heuristics}

The following subsections contain the heuristics we applied to prune the amount of possible actions
depending on the phase of the game.

\subsection{Set-Up Phase}
\label{subsec:set-up-phase}

We define the set-up phase as the phase in which players claim free territories at the beginning of the game.
Our strategy is to first conquer South America and then North America or Africa \cite{web:south-america}.
We therefore hard-coded the corresponding territory-ids into an array like an opening book.
The agent will then try to claim the first free country from that list.
We also check if the enemy is close to conquering a continent and occupy a territory to prevent any continent bonus.

\subsection{Reinforcement Phase}
\label{subsec:reinforcement phase}

Since the troop bonus for trading in cards is incremental,
we try to trade in as late as possible for the highest bonus \cite{web:1v1-risk-strategy}.
When setting troops, we only consider territories which have a neighboring territory occupied by the enemy.
We try to set more troops on fewer territories instead of spreading them out onto multiple countries.

\subsection{Attack Phase}
\label{subsec:attack-phase}

For the attack phase we found a table \ref{tab:expexted-troop-loss}
which contains the expected attacker troop loss per one defender unit.
Since it is always beneficial to attack when this number is lower than unity, we only keep those actions.
This also means that we attack territories which have more total troops than our attacking country \cite{osborne2003markov}.
Another optimization we used is to always attack with the maximum number of possible troops
since this will return higher odds of a win.

\begin{table}[htbp]
    \caption{Expected troop loss of the attacker \cite{web:per-attack-dice}.}
    \label{tab:expexted-troop-loss}
    \begin{center}
        \begin{tabular}{|l|c|c|c|c|c|c|}
            \hline
            Attacker & \multicolumn{2}{|c|}{1} & \multicolumn{2}{|c|}{2} & \multicolumn{2}{|c|}{3} \\
            \hline
            Defender & 1 & 2 & 1 & 2 & 1 & 2 \\
            \hline
            Expected Loss & 1.40 & 2.93 & 0.73 & 1.14 & 0.52 & 0.95 \\
            \hline
        \end{tabular}
    \end{center}
\end{table}

\subsection{Occupy Phase}
\label{subsec:occupy-phase}

Once a territory is conquered, we claim it with the maximum amount of troops in order to attack possible neighbors.
Since the attacker has the advantage (\ref{subsec:attack-phase}) we want keep attacking as long as possible.

\subsection{Fortify Phase}
\label{subsec:foritfy-phase}

In this phase we only move troops from a territory which has friendly neighbors exclusively.
Only countries with enemy neighbors can get these troops.
We always move with the maximum possible amount of troops.

\section{MCTS}
\label{sec:mcts}

We limit the number of simulations to 1000.
This was done by playing the agent against itself and staying under the predefined memory limit of 2~GB total
(1~GB per Agent) with the -Xmx option.

Contrary to conventional MCTS we do not let the game play out until the end in the simulation phase,
but to a maximum depth of 128 actions.
If the game does not end within the mentioned number of actions, we calculate a heuristic value
$h = \lfloor\# Territories/3\rfloor + continentBoni$
which is the amount of troops for the reinforcement phase.
Our testing showed that using this heuristic instead of complete play-out resulted in the agent winning in fewer actions.

\section{Further Optimizations}

There are a number of optimizations we wanted to implement but could not due to time constraints:
\begin{itemize}
    \item Calculate an attack path instead of attacking neighboring countries blindly.
    \item When trading in cards, prefer territories with enemy neighbors for two troops extra.
    \item Bias some nodes in the expansion phase instead of initializing them with zero plays and wins.
    \item Calculate front lines and only reinforce one territory per front line.
    \item Prefer 3v1 attacks before 3v2.
\end{itemize}

\bibliographystyle{unsrt}
\bibliography{references}

\end{document}
